% Options for packages loaded elsewhere
\PassOptionsToPackage{unicode}{hyperref}
\PassOptionsToPackage{hyphens}{url}
\PassOptionsToPackage{dvipsnames,svgnames,x11names}{xcolor}
%
\documentclass[
  letterpaper,
  DIV=11,
  numbers=noendperiod]{scrartcl}

\usepackage{amsmath,amssymb}
\usepackage{iftex}
\ifPDFTeX
  \usepackage[T1]{fontenc}
  \usepackage[utf8]{inputenc}
  \usepackage{textcomp} % provide euro and other symbols
\else % if luatex or xetex
  \usepackage{unicode-math}
  \defaultfontfeatures{Scale=MatchLowercase}
  \defaultfontfeatures[\rmfamily]{Ligatures=TeX,Scale=1}
\fi
\usepackage{lmodern}
\ifPDFTeX\else  
    % xetex/luatex font selection
\fi
% Use upquote if available, for straight quotes in verbatim environments
\IfFileExists{upquote.sty}{\usepackage{upquote}}{}
\IfFileExists{microtype.sty}{% use microtype if available
  \usepackage[]{microtype}
  \UseMicrotypeSet[protrusion]{basicmath} % disable protrusion for tt fonts
}{}
\makeatletter
\@ifundefined{KOMAClassName}{% if non-KOMA class
  \IfFileExists{parskip.sty}{%
    \usepackage{parskip}
  }{% else
    \setlength{\parindent}{0pt}
    \setlength{\parskip}{6pt plus 2pt minus 1pt}}
}{% if KOMA class
  \KOMAoptions{parskip=half}}
\makeatother
\usepackage{xcolor}
\setlength{\emergencystretch}{3em} % prevent overfull lines
\setcounter{secnumdepth}{5}
% Make \paragraph and \subparagraph free-standing
\ifx\paragraph\undefined\else
  \let\oldparagraph\paragraph
  \renewcommand{\paragraph}[1]{\oldparagraph{#1}\mbox{}}
\fi
\ifx\subparagraph\undefined\else
  \let\oldsubparagraph\subparagraph
  \renewcommand{\subparagraph}[1]{\oldsubparagraph{#1}\mbox{}}
\fi


\providecommand{\tightlist}{%
  \setlength{\itemsep}{0pt}\setlength{\parskip}{0pt}}\usepackage{longtable,booktabs,array}
\usepackage{calc} % for calculating minipage widths
% Correct order of tables after \paragraph or \subparagraph
\usepackage{etoolbox}
\makeatletter
\patchcmd\longtable{\par}{\if@noskipsec\mbox{}\fi\par}{}{}
\makeatother
% Allow footnotes in longtable head/foot
\IfFileExists{footnotehyper.sty}{\usepackage{footnotehyper}}{\usepackage{footnote}}
\makesavenoteenv{longtable}
\usepackage{graphicx}
\makeatletter
\def\maxwidth{\ifdim\Gin@nat@width>\linewidth\linewidth\else\Gin@nat@width\fi}
\def\maxheight{\ifdim\Gin@nat@height>\textheight\textheight\else\Gin@nat@height\fi}
\makeatother
% Scale images if necessary, so that they will not overflow the page
% margins by default, and it is still possible to overwrite the defaults
% using explicit options in \includegraphics[width, height, ...]{}
\setkeys{Gin}{width=\maxwidth,height=\maxheight,keepaspectratio}
% Set default figure placement to htbp
\makeatletter
\def\fps@figure{htbp}
\makeatother
% definitions for citeproc citations
\NewDocumentCommand\citeproctext{}{}
\NewDocumentCommand\citeproc{mm}{%
  \begingroup\def\citeproctext{#2}\cite{#1}\endgroup}
\makeatletter
 % allow citations to break across lines
 \let\@cite@ofmt\@firstofone
 % avoid brackets around text for \cite:
 \def\@biblabel#1{}
 \def\@cite#1#2{{#1\if@tempswa , #2\fi}}
\makeatother
\newlength{\cslhangindent}
\setlength{\cslhangindent}{1.5em}
\newlength{\csllabelwidth}
\setlength{\csllabelwidth}{3em}
\newenvironment{CSLReferences}[2] % #1 hanging-indent, #2 entry-spacing
 {\begin{list}{}{%
  \setlength{\itemindent}{0pt}
  \setlength{\leftmargin}{0pt}
  \setlength{\parsep}{0pt}
  % turn on hanging indent if param 1 is 1
  \ifodd #1
   \setlength{\leftmargin}{\cslhangindent}
   \setlength{\itemindent}{-1\cslhangindent}
  \fi
  % set entry spacing
  \setlength{\itemsep}{#2\baselineskip}}}
 {\end{list}}
\usepackage{calc}
\newcommand{\CSLBlock}[1]{\hfill\break\parbox[t]{\linewidth}{\strut\ignorespaces#1\strut}}
\newcommand{\CSLLeftMargin}[1]{\parbox[t]{\csllabelwidth}{\strut#1\strut}}
\newcommand{\CSLRightInline}[1]{\parbox[t]{\linewidth - \csllabelwidth}{\strut#1\strut}}
\newcommand{\CSLIndent}[1]{\hspace{\cslhangindent}#1}

\KOMAoption{captions}{tableheading}
\makeatletter
\@ifpackageloaded{tcolorbox}{}{\usepackage[skins,breakable]{tcolorbox}}
\@ifpackageloaded{fontawesome5}{}{\usepackage{fontawesome5}}
\definecolor{quarto-callout-color}{HTML}{909090}
\definecolor{quarto-callout-note-color}{HTML}{0758E5}
\definecolor{quarto-callout-important-color}{HTML}{CC1914}
\definecolor{quarto-callout-warning-color}{HTML}{EB9113}
\definecolor{quarto-callout-tip-color}{HTML}{00A047}
\definecolor{quarto-callout-caution-color}{HTML}{FC5300}
\definecolor{quarto-callout-color-frame}{HTML}{acacac}
\definecolor{quarto-callout-note-color-frame}{HTML}{4582ec}
\definecolor{quarto-callout-important-color-frame}{HTML}{d9534f}
\definecolor{quarto-callout-warning-color-frame}{HTML}{f0ad4e}
\definecolor{quarto-callout-tip-color-frame}{HTML}{02b875}
\definecolor{quarto-callout-caution-color-frame}{HTML}{fd7e14}
\makeatother
\makeatletter
\@ifpackageloaded{caption}{}{\usepackage{caption}}
\AtBeginDocument{%
\ifdefined\contentsname
  \renewcommand*\contentsname{Table of contents}
\else
  \newcommand\contentsname{Table of contents}
\fi
\ifdefined\listfigurename
  \renewcommand*\listfigurename{List of Figures}
\else
  \newcommand\listfigurename{List of Figures}
\fi
\ifdefined\listtablename
  \renewcommand*\listtablename{List of Tables}
\else
  \newcommand\listtablename{List of Tables}
\fi
\ifdefined\figurename
  \renewcommand*\figurename{Figure}
\else
  \newcommand\figurename{Figure}
\fi
\ifdefined\tablename
  \renewcommand*\tablename{Table}
\else
  \newcommand\tablename{Table}
\fi
}
\@ifpackageloaded{float}{}{\usepackage{float}}
\floatstyle{ruled}
\@ifundefined{c@chapter}{\newfloat{codelisting}{h}{lop}}{\newfloat{codelisting}{h}{lop}[chapter]}
\floatname{codelisting}{Listing}
\newcommand*\listoflistings{\listof{codelisting}{List of Listings}}
\makeatother
\makeatletter
\makeatother
\makeatletter
\@ifpackageloaded{caption}{}{\usepackage{caption}}
\@ifpackageloaded{subcaption}{}{\usepackage{subcaption}}
\makeatother
\ifLuaTeX
  \usepackage{selnolig}  % disable illegal ligatures
\fi
\usepackage{bookmark}

\IfFileExists{xurl.sty}{\usepackage{xurl}}{} % add URL line breaks if available
\urlstyle{same} % disable monospaced font for URLs
\hypersetup{
  pdftitle={Gender Disparities in Hiring \& Political Influence (U.S. and Global, 2024--2025)},
  pdfauthor={Pranathi Nagasai Andhe; Emily Sunderberg},
  colorlinks=true,
  linkcolor={blue},
  filecolor={Maroon},
  citecolor={Blue},
  urlcolor={Blue},
  pdfcreator={LaTeX via pandoc}}

\title{Gender Disparities in Hiring \& Political Influence (U.S. and
Global, 2024--2025)}
\author{Pranathi Nagasai Andhe \and Emily Sunderberg}
\date{2025-10-15}

\begin{document}
\maketitle

\renewcommand*\contentsname{Table of contents}
{
\hypersetup{linkcolor=}
\setcounter{tocdepth}{3}
\tableofcontents
}
\section{Abstract}\label{abstract}

We analyze gender disparities in hiring, AI participation, and wages
across U.S. industries and states, and situate these patterns in a
global context. Drawing on recent federal statistics ( (Anon. (2025a))),
U.S. pay-gap trackers (American Association of University Women
((2025a)); (Anon. (2024a))), and international reports ( (Anon.
(2024b)); (Anon. (2023)); (Anon. (2025b)); (Anon. (2025c))), we find
persistent occupational segregation, widening U.S. annual earnings gaps
since 2022, and continued underrepresentation of women in AI-intensive
roles. State-level disparities correlate with policy environments such
as pay-transparency and salary-history bans (American Association of
University Women ((2025b))), though industry composition and family
structure are important confounders. Conservative/market-based
perspectives emphasize occupational choice, hours, and career continuity
as key mechanisms and warn that some transparency policies may compress
wages (American Enterprise Institute ((2021)); Heritage Foundation
((2024)); Cullen and Pakzad-Hurson ((2021)); Cullen ((2023)); Mas
((2014))). We integrate both views and outline implications for job
seekers selecting sectors, geographies, and employers.

\section{Executive Summary}\label{executive-summary}

\begin{itemize}
\tightlist
\item
  U.S. employment remains gender-segregated: women ≈47\% of workers but
  overrepresented in health/education and underrepresented in
  construction, engineering, and many tech roles ( (Anon. (2025a))).
\item
  Every U.S. state has a pay gap; gaps tend to be smaller where
  transparency and equal-pay policies are stronger, though composition
  matters (American Association of University Women ((2025b))).
\item
  U.S. annual earnings gap widened after 2022 (82.7¢ in 2023; ≈80.9¢ in
  2024 among full-time year-round), while hourly measures show
  \textasciitilde85\% in 2024 ( (Anon. (2024a)); Pew Research Center
  ((2025))).
\item
  Women remain 22--30\% of the global AI workforce and are more exposed
  to AI-driven task change in clerical/admin roles ( (Anon. (2024b));
  (Anon. (2023)); (Anon. (2025b))).
\item
  Globally, parity stands at 68.8\% and may take \textasciitilde123
  years at current pace ( (Anon. (2025c))).
\item
  Market-oriented analyses attribute much of the unadjusted gap to
  hours, occupation, and career continuity and note possible wage
  compression from transparency (American Enterprise Institute ((2021));
  Heritage Foundation ((2024)); Cullen and Pakzad-Hurson ((2021));
  Cullen ((2023)); Mas ((2014))).
\item
  Implications: build AI-complementary skills, target transparent
  employers and supportive states/metros, and use posted pay bands as
  inputs to evidence-based negotiation.
\end{itemize}

\section{Introduction}\label{introduction}

Gender continues to shape labor-market outcomes in the United States and
worldwide. In 2024--2025, women's representation varies sharply across
industries, wage gaps persist, and AI both creates opportunities and
raises exposure risks. We examine four questions: (1) how hiring
patterns differ for men and women across industries; (2) whether
disparities vary between red and blue states; (3) whether women are more
underrepresented in AI fields; and (4) how wage gaps compare by gender
and political affiliations. We synthesize high-quality, recent
statistics and research to inform job-seeker strategy.

\section{Qualitative Research Method}\label{qualitative-research-method}

We triangulate multiple sources: U.S. Bureau of Labor Statistics Current
Population Survey (CPS) 2024 annual averages for occupational
distributions ( (Anon. (2025a))); AAUW 2025 national and state pay-gap
indicators (American Association of University Women ((2025a)); American
Association of University Women ((2025b))); IWPR 2024 fact sheets on
annual earnings gaps ( (Anon. (2024a))); Pew Research Center 2025 hourly
pay-gap analysis (Pew Research Center ((2025))); OECD 2024 policy brief
on AI and women ( (Anon. (2024b))); International Labour Organization
reports on generative-AI exposure (2023; 2025 update) ( (Anon. (2023));
(Anon. (2025b))); and the WEF 2025 Global Gender Gap Report ( (Anon.
(2025c))). To incorporate conservative/market perspectives, we review
AEI and Heritage commentary (American Enterprise Institute ((2021));
Heritage Foundation ((2024))) and research on equilibrium effects of pay
transparency (Cullen and Pakzad-Hurson ((2021)); Cullen ((2023)); Mas
((2014))).

\section{Hiring Patterns Across Industries (U.S. and
Global)}\label{hiring-patterns-across-industries-u.s.-and-global}

U.S. employment remains gender-segregated. CPS 2024 annual averages
indicate women comprise roughly 47\% of total employment but are more
concentrated in health care, education, and service roles, with lower
shares in construction, engineering, and portions of tech ( (Anon.
(2025a))). Internationally, the World Economic Forum (2025) estimates
overall global gender parity at 68.8\%, with economic participation
parity at about 60--61\%, implying persistent cross-country segmentation
( (Anon. (2025c))). Market-oriented analyses argue that part of observed
differences in outcomes reflect hours worked, occupation mix, and career
continuity rather than like-for-like pay differences (American
Enterprise Institute ((2021)); Heritage Foundation ((2024))).

\section{State Politics and Gender Disparities (Red
vs.~Blue)}\label{state-politics-and-gender-disparities-red-vs.-blue}

AAUW's 2025 analysis shows that every U.S. state has a gender pay gap,
with substantial dispersion across states (American Association of
University Women ((2025b))). Cross-state differences correlate with
policy adoption such as salary-history bans and pay-transparency
requirements, which are more prevalent in many blue states (American
Association of University Women ((2025b))). However, composition
matters: industry mix (e.g., energy and construction), unionization,
urbanization, and childcare access vary across states and can generate
red--blue patterns without ideology being the sole driver. Earlier
peer-reviewed work associates state liberalism with narrower gaps, but
causality remains difficult to establish (Maume ((2015))). Recent
reporting suggests that post-pandemic return-to-office mandates have
reduced flexibility and may contribute to widening national gaps, though
these effects likely differ by state and sector (The Washington Post
((2025))).

\subsection{Embedded visualization: AAUW live state
map}\label{embedded-visualization-aauw-live-state-map}

\begin{tcolorbox}[enhanced jigsaw, breakable, bottomtitle=1mm, coltitle=black, colbacktitle=quarto-callout-note-color!10!white, left=2mm, bottomrule=.15mm, colframe=quarto-callout-note-color-frame, arc=.35mm, colback=white, rightrule=.15mm, opacityback=0, toprule=.15mm, title=\textcolor{quarto-callout-note-color}{\faInfo}\hspace{0.5em}{Note}, toptitle=1mm, opacitybacktitle=0.6, leftrule=.75mm, titlerule=0mm]

\textbf{Tip:} The live embed is perfect for your qualitative section.
For PDF exports, also include a static image below as fallback.

\end{tcolorbox}

\emph{Source: American Association of University Women ((2025b)).}

\section{Women in AI Fields (U.S. \& Global) and Political
Context}\label{women-in-ai-fields-u.s.-global-and-political-context}

Women remain underrepresented in AI and tech roles. The OECD documents
lower female representation in AI-exposed professional occupations and
constrained access to AI tools ( (Anon. (2024b))). The ILO shows that
clerical and administrative tasks---female-heavy---are highly exposed to
generative-AI transformation in high-income countries; a 2025 refinement
confirms the asymmetric exposure ( (Anon. (2023)); (Anon. (2025b))).
Direct state-by-state measures of female AI participation are limited.
It is therefore premature to assert causality from political ideology to
AI underrepresentation without merging employer-level AI job postings
and hires with state policy and industry controls. Nonetheless,
differences in STEM pipelines, childcare, higher education, and
transparency regimes plausibly contribute to cross-state variation (
(Anon. (2023)); (Anon. (2025b)); (Anon. (2024b))).

\section{Wage Gaps and Political
Affiliation}\label{wage-gaps-and-political-affiliation}

On annual full-time, year-round earnings, IWPR reports a deterioration
from 2022 to 2023 (82.7¢) and news coverage indicates about 80.9¢ in
2024---the lowest since 2016 ( (Anon. (2024a)); Newsweek ((2025))). By
contrast, Pew Research Center's hourly series shows women earned about
85\% of men's hourly pay in 2024 when combining full- and part-time
workers (Pew Research Center ((2025))). Adjusted gaps shrink after
controlling for occupation, hours, and experience but do not disappear
(American Enterprise Institute ((2021)); Pew Research Center ((2025))).
Policy can narrow gaps: pay-transparency laws are associated with
smaller within-firm gaps but may compress overall wages or slow wage
growth according to equilibrium analyses (Cullen and Pakzad-Hurson
((2021)); Cullen ((2023)); Mas ((2014))).

\section{Implications for Job Seekers
(2025)}\label{implications-for-job-seekers-2025}

\begin{itemize}
\tightlist
\item
  \textbf{Sector choice:} Target underrepresented, higher-growth fields
  such as data, AI, and engineering while building verifiable skills,
  certifications, and portfolios ( (Anon. (2025a)); (Anon. (2024b))).
\item
  \textbf{Geography:} Favor states and metros with pay-transparency
  requirements and supportive care infrastructure while benchmarking
  offers with state snapshots (American Association of University Women
  ((2025b))).
\item
  \textbf{AI resilience:} Develop AI-complementary skills to hedge
  exposure in clerical/admin roles and to compete for AI-adjacent,
  higher-pay tracks ( (Anon. (2023)); (Anon. (2024b))).
\item
  \textbf{Employer screening:} Prefer organizations with posted pay
  bands, career-progression transparency, and flexible/hybrid policies
  as RTO mandates may widen disparities (The Washington Post ((2025))).
\item
  \textbf{Negotiation:} Use posted ranges as inputs, not anchors, and
  negotiate based on documented contributions; be aware of
  transparency's potential compression effects (Cullen and Pakzad-Hurson
  ((2021)); Cullen ((2023)); Mas ((2014))).
\end{itemize}

\section{Limitations}\label{limitations}

Causal attribution of political ideology to gender disparities is
challenging due to confounding by industry mix, demographics, and local
cost structures. AI participation statistics with state-gender
granularity remain sparse. International comparisons depend on differing
definitions of occupations, pay, and employment. Transparency policy
effects vary by market and occupation; equilibrium responses may offset
some intended benefits.

\section{Conclusion}\label{conclusion}

Gender disparities in hiring, AI participation, and pay persist across
the U.S. and globally. State policies and employer practices shape
observed gaps, but composition and choice also matter. A pragmatic
job-search strategy in 2025 combines sector targeting, AI-adjacent
upskilling, careful geography selection, and screening for transparent,
flexible employers. Continuous measurement using CPS updates, AAUW/IWPR
dashboards, and international benchmarks will be essential for tracking
progress.

\phantomsection\label{refs}
\begin{CSLReferences}{1}{1}
\bibitem[\citeproctext]{ref-aauw2025simpletruth}
\textsc{American Association of University Women}. (2025a):
{``\href{https://www.aauw.org/resources/research/simple-truth/}{The
Simple Truth about the Gender Pay Gap 2025},''}

\bibitem[\citeproctext]{ref-aauw2025state}
\textsc{-\/-\/-}. (2025b):
{``\href{https://www.aauw.org/resources/article/gender-pay-gap-by-state/}{Gender
Pay Gap by State},''}

\bibitem[\citeproctext]{ref-aei}
\textsc{American Enterprise Institute}. (2021):
{``\href{https://www.aei.org/}{American Enterprise Institute --- Gender
Pay Gap Commentary Archive},''}

\bibitem[\citeproctext]{ref-ilo2023genai}
\textsc{Anon.} (2023):
\href{https://www.ilo.org/sites/default/files/wcmsp5/groups/public/\%40dgreports/\%40inst/documents/publication/wcms_890761.pdf}{Generative
AI and Jobs: A Global Analysis of Potential Effects},International
Labour Organization.

\bibitem[\citeproctext]{ref-iwpr2024gap}
\textsc{-\/-\/-Anon.} (2024a):
\href{https://iwpr.org/wp-content/uploads/2024/09/IWPR-National-Wage-Gap-Fact-Sheet-2024.pdf}{Gender
and Racial Wage Gaps Worsened in 2023 and Pay Equity Still Decades
Away},Institute for Women's Policy Research.

\bibitem[\citeproctext]{ref-oecd2024algorithm}
\textsc{-\/-\/-Anon.} (2024b):
\href{https://www.oecd.org/content/dam/oecd/en/publications/reports/2024/12/algorithm-and-eve_0e889c45/a1603510-en.pdf}{Algorithm
and Eve: How AI Will Impact Women at Work},OECD.

\bibitem[\citeproctext]{ref-bls2025cps}
\textsc{-\/-\/-Anon.} (2025a):
\href{https://www.bls.gov/cps/cpsaat11.htm}{CPS Table a-11: Employed
Persons by Detailed Occupation, Sex, Race, and Hispanic or Latino
Ethnicity (Annual Averages, 2024)},U.S. Bureau of Labor Statistics.

\bibitem[\citeproctext]{ref-ilo2025exposure}
\textsc{-\/-\/-Anon.} (2025b):
\href{https://www.ilo.org/publications/generative-ai-and-jobs-refined-global-index-occupational-exposure}{Generative
AI and Jobs: A Refined Global Index of Occupational
Exposure},International Labour Organization.

\bibitem[\citeproctext]{ref-wef2025ggg}
\textsc{-\/-\/-Anon.} (2025c):
\href{https://reports.weforum.org/docs/WEF_GGGR_2025.pdf}{Global Gender
Gap Report 2025},World Economic Forum.

\bibitem[\citeproctext]{ref-cullen2023}
\textsc{Cullen, Z.} (2023):
\href{https://www.nber.org/system/files/working_papers/w31060/w31060.pdf}{Is
Pay Transparency Good?},National Bureau of Economic Research.

\bibitem[\citeproctext]{ref-cullen2021}
\textsc{Cullen, Z., and B. Pakzad-Hurson}. (2021):
\href{https://www.nber.org/system/files/working_papers/w28903/revisions/w28903.rev0.pdf}{Equilibrium
Effects of Pay Transparency},National Bureau of Economic Research.

\bibitem[\citeproctext]{ref-heritage}
\textsc{Heritage Foundation}. (2024):
{``\href{https://www.heritage.org/jobs-and-labor/commentary/making-sense-the-wage-gap}{Making
Sense of the Wage Gap},''}

\bibitem[\citeproctext]{ref-mas2014transparency}
\textsc{Mas, A.} (2014):
{``\href{https://www.nber.org/system/files/working_papers/w20558/w20558.pdf}{Does
transparency lead to pay compression?}''}

\bibitem[\citeproctext]{ref-maume2015state}
\textsc{Maume, D. J.} (2015): {``State liberalism, female supervisors,
and the gender wage gap,''} \emph{Social Science Research},.

\bibitem[\citeproctext]{ref-newsweek2025gap}
\textsc{Newsweek}. (2025):
{``\href{https://www.newsweek.com/americas-gender-pay-gap-going-wrong-direction-10739408}{America's
Gender Pay Gap Going in Wrong Direction},''}

\bibitem[\citeproctext]{ref-pew2025gap}
\textsc{Pew Research Center}. (2025):
{``\href{https://www.pewresearch.org/short-reads/2025/03/04/gender-pay-gap-in-us-has-narrowed-slightly-over-2-decades/}{Gender
Pay Gap in u.s. Has Narrowed Slightly over 2 Decades},''}

\bibitem[\citeproctext]{ref-washingtonpost2025rto}
\textsc{The Washington Post}. (2025):
{``\href{https://www.washingtonpost.com/business/2025/10/11/rto-mandates-gender-wage-gap/}{Women
Are Taking Pay Cuts as Companies Mandate Return to Office},''}

\end{CSLReferences}



\end{document}
